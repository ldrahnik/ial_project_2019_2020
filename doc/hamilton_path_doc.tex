\documentclass[a4paper,11pt]{article}

\usepackage[left=20mm, text={170mm, 240mm}, top=30mm]{geometry}
\usepackage[czech]{babel}
\usepackage[IL2]{fontenc}
\usepackage[utf8x]{inputenc}
\usepackage{enumitem}
\usepackage{scrextend}
\usepackage{lscape}
\usepackage{times}
\usepackage{graphicx}
\usepackage[T1]{fontenc}
\usepackage{lmodern}
\usepackage{indentfirst}
\usepackage{pgfplots}
\usepackage{pgfplotstable}
\usepackage{mathtools}
\usepackage{amsfonts}
\usepackage{amsthm}
\usepackage{booktabs}

\setlength{\parskip}{1em}
\pgfplotsset{compat=1.15}

\pgfplotstableset{% global config, for example in the preamble
  every head row/.style={before row=\toprule,after row=\midrule},
  every last row/.style={after row=\bottomrule},
  fixed
}

\begin{document}

\begin{titlepage}

	\begin{center}
		{\Huge\textsc{Vysoké učení technické v~Brně}}\\
		\medskip
		{\huge\textsc{Fakulta informačních technologií}}\\
		\vspace{\stretch{0.382}}
		{\huge 2. Hamiltonova cesta a cyklus v grafu}\\
		\medskip
		{\LARGE IAL - Algoritmy: Náhradní projekt - skupinový}\\
		\vspace{\stretch{0.618}}
	\end{center}

    \noindent xdrahn00@stud.fit.vutbr.cz \Large {\hfill Brno, \today}

\end{titlepage}

\section{Úvod}

Projekt do

\section{Zadání}

Cestu v grafu, ve které se vyskytuje každý vrchol právě jednou, nazýváme Hamiltonovou cestou. Má-li tato cesta počátek a konec v jednom jediném vrcholu, pak se jedná o Hamiltonův cyklus v grafu. 

Vytvořte program pro hledání Hamiltonovy cesty (pro dva zadané vrcholy) a Hamiltonova cyklu v neorientovaném grafu. 

Pokud existuje více řešení, nalezněte všechna. Výsledky prezentujte vhodným způsobem. Součástí projektu bude načítání grafů ze souboru a vhodné testovací grafy. V dokumentaci uveďte teoretickou složitost úlohy a porovnejte ji s experimentálními výsledky.

test citate \cite{test_citate}

\pgfplotstabletypeset[
  columns/graph_name/.style={string type, column name=Název grafu},
  columns/vertices_count/.style={column name=Počet vrcholů},
  columns/edges_count/.style={column name=Počet hran},
  columns/explored_vertices_count/.style={column name=Počet prozkoumaných vrcholů},
  columns/executed_time/.style={column name=Čas běhu programu [s]},
]{./../time_coplexity/brute_force_info.txt}

\begin{figure}[h!]
\centering
\begin{tikzpicture}
\begin{axis}[
    xlabel={Počet vrcholů grafu (hrany každý s každým)},
    ylabel={Počet prozkoumaných vrcholů},
    legend pos=north west,
    legend entries={Brute force algoritmus},
]
\addplot table [x=vertices_count,y=explored_vertices_count] {./../time_coplexity/brute_force_info.txt};
\end{axis}
\end{tikzpicture}
\end{figure}

\begin{figure}[h!]
\centering
\begin{tikzpicture}
\begin{axis}[
    xlabel={Počet vrcholů grafu (hrany každý s každým)},
    ylabel={Čas běhu programu [s]},
    legend pos=north west,
    legend entries={Brute force algoritmus},
]
\addplot table [x=vertices_count,y=executed_time] {./../time_coplexity/brute_force_info.txt};
\end{axis}
\end{tikzpicture}
\end{figure}

\newpage

\bibliographystyle{czechiso}

\bibliography{bibliography}

\end{document}
